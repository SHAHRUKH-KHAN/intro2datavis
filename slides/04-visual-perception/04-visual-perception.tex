\documentclass[12pt]{beamer}\usepackage[]{graphicx}\usepackage[]{color}
%% maxwidth is the original width if it is less than linewidth
%% otherwise use linewidth (to make sure the graphics do not exceed the margin)
\makeatletter
\def\maxwidth{ %
  \ifdim\Gin@nat@width>\linewidth
    \linewidth
  \else
    \Gin@nat@width
  \fi
}
\makeatother

\definecolor{fgcolor}{rgb}{0.345, 0.345, 0.345}
\newcommand{\hlnum}[1]{\textcolor[rgb]{0.686,0.059,0.569}{#1}}%
\newcommand{\hlstr}[1]{\textcolor[rgb]{0.192,0.494,0.8}{#1}}%
\newcommand{\hlcom}[1]{\textcolor[rgb]{0.678,0.584,0.686}{\textit{#1}}}%
\newcommand{\hlopt}[1]{\textcolor[rgb]{0,0,0}{#1}}%
\newcommand{\hlstd}[1]{\textcolor[rgb]{0.345,0.345,0.345}{#1}}%
\newcommand{\hlkwa}[1]{\textcolor[rgb]{0.161,0.373,0.58}{\textbf{#1}}}%
\newcommand{\hlkwb}[1]{\textcolor[rgb]{0.69,0.353,0.396}{#1}}%
\newcommand{\hlkwc}[1]{\textcolor[rgb]{0.333,0.667,0.333}{#1}}%
\newcommand{\hlkwd}[1]{\textcolor[rgb]{0.737,0.353,0.396}{\textbf{#1}}}%
\let\hlipl\hlkwb

\usepackage{framed}
\makeatletter
\newenvironment{kframe}{%
 \def\at@end@of@kframe{}%
 \ifinner\ifhmode%
  \def\at@end@of@kframe{\end{minipage}}%
  \begin{minipage}{\columnwidth}%
 \fi\fi%
 \def\FrameCommand##1{\hskip\@totalleftmargin \hskip-\fboxsep
 \colorbox{shadecolor}{##1}\hskip-\fboxsep
     % There is no \\@totalrightmargin, so:
     \hskip-\linewidth \hskip-\@totalleftmargin \hskip\columnwidth}%
 \MakeFramed {\advance\hsize-\width
   \@totalleftmargin\z@ \linewidth\hsize
   \@setminipage}}%
 {\par\unskip\endMakeFramed%
 \at@end@of@kframe}
\makeatother

\definecolor{shadecolor}{rgb}{.97, .97, .97}
\definecolor{messagecolor}{rgb}{0, 0, 0}
\definecolor{warningcolor}{rgb}{1, 0, 1}
\definecolor{errorcolor}{rgb}{1, 0, 0}
\newenvironment{knitrout}{}{} % an empty environment to be redefined in TeX

\usepackage{alltt}
\usepackage{tikz}

% make it pretty
% get rid of junk
\usetheme{default}
\usefonttheme[onlymath]{serif}
\beamertemplatenavigationsymbolsempty

% define a bunch of colors
\definecolor{offwhite}{RGB}{255,250,240}
\definecolor{gray}{RGB}{155,155,155}
\definecolor{foreground}{RGB}{80,80,80}
\definecolor{background}{RGB}{255,255,255}
%\definecolor{title}{RGB}{255,199,0}
\definecolor{title}{RGB}{89,132,212}
%\definecolor{subtitle}{RGB}{89,132,212}
\definecolor{subtitle}{RGB}{255,199,0}
\definecolor{hilit}{RGB}{248,117,79}
\definecolor{vhilit}{RGB}{255,111,207}
\definecolor{lolit}{RGB}{200,200,200}
\definecolor{lit}{RGB}{255,199,0}
\definecolor{mdlit}{RGB}{89,132,212}
\definecolor{link}{RGB}{248,117,79}

% a few color macros
\newcommand{\hilit}{\color{hilit}}
\newcommand{\vhilit}{\color{vhilit}}
\newcommand{\lit}{\color{lit}}
\newcommand{\mdlit}{\color{mdlit}}
\newcommand{\lolit}{\color{lolit}}

% use those colors
\setbeamercolor{titlelike}{fg=title}
\setbeamercolor{subtitle}{fg=subtitle}
\setbeamercolor{frametitle}{fg=gray}
%\setbeamercolor{structure}{fg=subtitle}
\setbeamercolor{structure}{fg=title}
\setbeamercolor{institute}{fg=lolit}
\setbeamercolor{normal text}{fg=foreground,bg=background}
\setbeamertemplate{itemize subitem}{{\textendash}}
\setbeamerfont{itemize/enumerate subbody}{size=\small}
\setbeamerfont{itemize/enumerate subitem}{size=\small}

% center title of slides
\setbeamertemplate{blocks}[rounded]
\setbeamertemplate{frametitle}[default][center]

% page number
\setbeamerfont{page number in foot}{size=\footnotesize}
\setbeamertemplate{footline}[frame number]

% default link color
\hypersetup{colorlinks, urlcolor={link}}

% a few macros
\newcommand{\code}[1]{\texttt{#1}}
\newcommand{\hicode}[1]{{\hilit \texttt{#1}}}
\newcommand{\locode}[1]{{\lolit \texttt{#1}}}
\newcommand{\bb}[1]{\begin{block}{#1}}
\newcommand{\eb}{\end{block}}
\newcommand{\bi}{\begin{itemize}}
\newcommand{\bbi}{\vspace{4pt} \begin{itemize} \itemsep8pt}
\newcommand{\ei}{\end{itemize}}
\newcommand{\bv}{\begin{verbatim}}
\newcommand{\ev}{\end{verbatim}}
\newcommand{\ig}{\includegraphics}
\newcommand{\subt}[1]{{\footnotesize \color{subtitle} {#1}}}
\newcommand{\ttsm}{\tt \small}
\newcommand{\ttfn}{\tt \footnotesize}
\newcommand{\figh}[2]{\centerline{\includegraphics[height=#2\textheight]{#1}}}
\newcommand{\figw}[2]{\centerline{\includegraphics[width=#2\textwidth]{#1}}}



%------------------------------------------------

\title{Visual Perception}
\subtitle{Intro to Data Visualization}
\author{\href{http://www.gastonsanchez.com}{Gaston Sanchez}}
\institute{\href{https://creativecommons.org/licenses/by-sa/4.0/}{\tt \scriptsize \color{foreground} CC BY-SA 4.0}}
\date{}
\IfFileExists{upquote.sty}{\usepackage{upquote}}{}
\begin{document}

% no page number in first slide
{
  \setbeamertemplate{footline}{} 
  \frame{\titlepage} 
}

%------------------------------------------------

\begin{frame}
\begin{center}
\Huge{\hilit{Visual Perception}}
\end{center}
\end{frame}

%------------------------------------------------

\begin{frame}
\frametitle{Visual Brain}
\begin{center}
\ig[width=11cm]{images/psychophysical-system1.pdf}
\end{center}
\end{frame}
%------------------------------------------------

\begin{frame}
\frametitle{What is Perception?}

\bbi
  \item It's a cognitive process.
  \item Involves interpretation of the world around us.
  \item Allows us to form mental representations of the environment.
  \item Our brain makes assumptions about the world to overcome the inherent ambiguity
in all sensory data.
\ei

\end{frame}

%------------------------------------------------

\begin{frame}
\frametitle{Many definitions and theories of perception}

\bb{Most theories define perception as the process of:}
\bbi
  \item \textbf{Recognizing} (being aware of) sensory information
  \item \textbf{Organizing} (gathering and storing) sensory information
  \item \textbf{Interpreting} (binding to knowledge) sensory information
\ei
\eb

\end{frame}

%------------------------------------------------

\begin{frame}
\frametitle{Visual Brain}
\begin{center}
\ig[width=11cm]{images/psychophysical-system2.pdf}
\end{center}
\end{frame}

%------------------------------------------------

\begin{frame}
\frametitle{Brain as a computer}

\bb{Brain as a computer}
To better understand the process of visual information, it is useful 
to think about the brain as a computer.
\eb

\end{frame}

%------------------------------------------------

\begin{frame}
\begin{center}
\Huge{\hilit{Attention and Memory}}
\end{center}
\end{frame}

%------------------------------------------------

\begin{frame}
\frametitle{Information Processing and Memory}
\begin{center}
\ig[width=11cm]{images/visual-brain-stages.pdf}
\end{center}
\end{frame}

%------------------------------------------------

\begin{frame}
\frametitle{The brain as a computer}

\bb{Types of memory for processing visual information}
\bbi
  \item Iconic memory (visual sensory register) \\
  {\lolit like the buffer or temporary}
  \item Short-term memory (working memory) \\
  {\lolit like the random access memory (RAM)}
  \item Long-term memory (``permanent'' storage) \\
  {\lolit like the hard disk}
\ei
\eb

\end{frame}

%------------------------------------------------

\begin{frame}
\frametitle{Iconic Memory}
\begin{center}
\ig[width=11cm]{images/memory-iconic.pdf}
\end{center}
\end{frame}

%------------------------------------------------

\begin{frame}
\frametitle{Iconic Memory}

\bbi
  \item The \textbf{iconic memory} is a sort of waiting room where each snapshot of
  input waits to be passed on to short-term memory.
  \item Rapid processing: almost automatic and unconscious.
  \item Also called \textbf{preattentive processing}.
  \item Extremely fast and parallel processing.
  \item Processes primitive visual features.
\ei

\end{frame}

%------------------------------------------------

\begin{frame}
\frametitle{Short-term Memory}
\begin{center}
\ig[width=11cm]{images/memory-short-term.pdf}
\end{center}
\end{frame}

%------------------------------------------------

\begin{frame}
\frametitle{Short-term Memory}

\bbi
  \item The short-term memory is a sort of RAM.
  \item This is where conscious mental work is performed to support cognition.
  \item Information is combined into meaningful visual chunks.
  \item This memory is temporary and has limited storage capacity.
  \item Where the attentive process of perception occurs.
\ei

\end{frame}

%------------------------------------------------

\begin{frame}
\frametitle{Long-term Memory}
\begin{center}
\ig[width=11cm]{images/memory-long-term.pdf}
\end{center}
\end{frame}

%------------------------------------------------

\begin{frame}
\frametitle{Long-term Memory}

\bbi
  \item The long-term memory is a sort of hard disk.
  \item It's a dynamic structure that retains everything we know.
  \item Involves an intricate network of links and cross-references that help
  us find information.
  \item Holds our ability to recognize images and detect meaningful patterns.
  \item When you selectively pay attention to information in working memory,
  it is likely to get transformed and encoded into long-term memory.
\ei

\end{frame}

%------------------------------------------------

\begin{frame}
\frametitle{Processing Visual Information}
\begin{center}
\ig[width=11cm]{images/visual-brain.pdf}
\end{center}
\end{frame}

%------------------------------------------------

\begin{frame}
\frametitle{3 Stages of perceptual processing}

\bbi
  \item \textit{Iconic memory}: early, parallel detection of color, texture, 
  shape, and spatial attributes.
  \item \textit{Short-term memory}: dividing visual fields into regions and 
  simple patterns.
  \item \textit{Working-memory}: holding objects in working memory by demands 
  of active attention.
\ei

\end{frame}

%------------------------------------------------

\begin{frame}
\begin{center}
\Huge{\hilit{Illusionss}}
\end{center}
\end{frame}

%------------------------------------------------

\begin{frame}
\frametitle{Visual Perception and illusions}

Visual representations of objects are often misinterpreted.

\bi
  \item Illusions are a primary source of misinterpretations.
  \item They come in a variety of forms.
  \item Common geometric illusions are:
  \bi
    \item distortions of lengths
    \item distortions of angles
    \item distortions of areas
    \item distortions of shapes
  \ei
\ei

\end{frame}

%------------------------------------------------

\begin{frame}
\frametitle{Ponzo illusion}
\begin{center}
\ig[height=5cm]{images/illusion-ponzo.pdf}
\end{center}
\end{frame}

%------------------------------------------------

\begin{frame}
\frametitle{Poggendorf illusion}
\begin{center}
\ig[width=8cm]{images/illusion-poggendorf1.pdf}
\end{center}
\end{frame}

%------------------------------------------------

\begin{frame}
\frametitle{Poggendorf illusion}
\begin{center}
\ig[width=8cm]{images/illusion-poggendorf2.pdf}
\end{center}
\end{frame}

%------------------------------------------------

\begin{frame}
\frametitle{Zollner illusion}
\begin{center}
\ig[height=6cm]{images/illusion-zollner1.pdf}
\end{center}
\end{frame}

%------------------------------------------------

\begin{frame}
\frametitle{Zollner illusion}
\begin{center}
\ig[height=6cm]{images/illusion-zollner2.pdf}
\end{center}
\end{frame}

%------------------------------------------------

\begin{frame}
\frametitle{Tichener illusion}
\begin{center}
\ig[width=7cm]{images/illusion-tichener.pdf}
\end{center}
\end{frame}

%------------------------------------------------

\begin{frame}
\frametitle{A Line Length illusion}
\begin{center}
\ig[width=5cm]{images/illusion-line-length.pdf}
\end{center}
\end{frame}

%------------------------------------------------

\begin{frame}
\frametitle{Perspective illusion}
\begin{center}
\ig[height=7cm]{images/illusion-perspective.pdf}
\end{center}
\end{frame}

%------------------------------------------------

\begin{frame}
\frametitle{The Herman Grid}
\begin{center}
\ig[height=6cm]{images/illusion-herman-grid1.pdf}
\end{center}
\end{frame}

%------------------------------------------------

\begin{frame}
\frametitle{The Herman Grid}
\begin{center}
\ig[height=6cm]{images/illusion-herman-grid2.pdf}
\end{center}
\end{frame}

%------------------------------------------------

\begin{frame}
\frametitle{Kanizsa illusion}
\begin{center}
\ig[height=5cm]{images/illusion-kanizsa.pdf}
\end{center}
\end{frame}

%------------------------------------------------

\begin{frame}
\frametitle{More illusions}
\begin{center}
\ig[height=7cm]{images/illusion-rotating-wheels.pdf}
\end{center}
\end{frame}

%------------------------------------------------

\begin{frame}
\frametitle{Kanizsa Illusion}
\begin{center}
\ig[height=5cm]{images/illusion-kanizsa.pdf}
\end{center}
\end{frame}

%------------------------------------------------

\begin{frame}
\begin{center}
\Huge{\hilit{Perceptual Processing}}
\end{center}
\end{frame}

%------------------------------------------------

\begin{frame}
\frametitle{Perceptual Processing}

\bbi
  \item Classic model of information processing
  \item Simple model for understanding the flow of sensory information
  \item Processing divided into preattentive and postattentive
\ei

\end{frame}

%------------------------------------------------

\begin{frame}
\frametitle{Preattentive Processing}

\bb{About Preattentive Processing}
  Researchers have discovered a limited set of individual properties that 
  are detected very rapidly and accurately by the low-level visual system.
  These properties were initially called \textit{preattentive}. 
  We now know that attention plays a critical role in what we see.
  The term \textit{preattentive} continues to be used
  Preattentive tasks are those performed in less than 200 to 250 milliseconds.
\eb

\end{frame}

%------------------------------------------------

\begin{frame}
\frametitle{Early Vision -vs- Later Vision}
\begin{center}
\ig[height=6cm]{images/early-later-vision.pdf}
\end{center}
\end{frame}

%------------------------------------------------

\begin{frame}
\begin{center}
\Huge{\hilit{Early Vision and \\ Preattentive Processing}}
\end{center}
\end{frame}

%------------------------------------------------

\begin{frame}
\frametitle{Stage 1: Low-level Preattentive Processing}

\bbi
  \item Arrays of neurons work in parallel
  \item Requires attention despite the name
  \item Occurs almost automatically (very fast: $<$ 200-250 ms)
  \item Information is transitory, briefly held in iconic store
  \item What matters most is the contrast between features
\ei

\end{frame}

%------------------------------------------------

\begin{frame}
\frametitle{Preattentive Processing}

\bb{Preattentive Features}
Visual features have been identified as preattentive in 
various experiments in psychology to perform 4 major visual tasks:
\bi
  \item target detection
  \item boundary detection
  \item region tracking
  \item counting and estimation
\ei
\eb

\end{frame}

%------------------------------------------------

\begin{frame}
\frametitle{Orientation}
\begin{center}
\ig[height=7cm]{images/preattentive-orientation.pdf}
\end{center}
\end{frame}

%------------------------------------------------

\begin{frame}
\frametitle{Length-Width}
\begin{center}
\ig[height=7cm]{images/preattentive-length-width.pdf}
\end{center}
\end{frame}

%------------------------------------------------

\begin{frame}
\frametitle{Closure}
\begin{center}
\ig[height=7cm]{images/preattentive-closure.pdf}
\end{center}
\end{frame}

%------------------------------------------------

\begin{frame}
\frametitle{Size}
\begin{center}
\ig[height=7cm]{images/preattentive-size.pdf}
\end{center}
\end{frame}

%------------------------------------------------

\begin{frame}
\frametitle{Curvature}
\begin{center}
\ig[height=7cm]{images/preattentive-curvature.pdf}
\end{center}
\end{frame}

%------------------------------------------------

\begin{frame}
\frametitle{Density}
\begin{center}
\ig[height=7cm]{images/preattentive-density.pdf}
\end{center}
\end{frame}

%------------------------------------------------

\begin{frame}
\frametitle{Number Estimation}
\begin{center}
\ig[height=7cm]{images/preattentive-number.pdf}
\end{center}
\end{frame}

%------------------------------------------------

\begin{frame}
\frametitle{Number Color (hue)}
\begin{center}
\ig[height=7cm]{images/preattentive-hue.pdf}
\end{center}
\end{frame}

%------------------------------------------------

\begin{frame}
\frametitle{Intensity}
\begin{center}
\ig[height=7cm]{images/preattentive-intensity.pdf}
\end{center}
\end{frame}

%------------------------------------------------

\begin{frame}
\frametitle{Intersection}
\begin{center}
\ig[height=7cm]{images/preattentive-intersection.pdf}
\end{center}
\end{frame}

%------------------------------------------------

\begin{frame}
\frametitle{Terminators}
\begin{center}
\ig[height=7cm]{images/preattentive-terminators.pdf}
\end{center}
\end{frame}

%------------------------------------------------

\begin{frame}
\frametitle{Simultaneous contrast}
\begin{center}
\ig[height=6cm]{images/simultaneous-contrast.pdf}
\end{center}
\end{frame}

%------------------------------------------------

\begin{frame}
\frametitle{How many 5s?}
\begin{center}
\ig[width=11cm]{images/how-many-fives1.pdf}
\end{center}
\end{frame}

%------------------------------------------------

\begin{frame}
\frametitle{How many 5s?}
\begin{center}
\ig[width=11cm]{images/how-many-fives2.pdf}
\end{center}
\end{frame}

%------------------------------------------------

\begin{frame}
\frametitle{Color}
\begin{center}
\ig[width=10cm]{images/preattentive-color.pdf}
\end{center}
\end{frame}

%------------------------------------------------

\begin{frame}
\frametitle{Shape}
\begin{center}
\ig[width=10cm]{images/preattentive-shape.pdf}
\end{center}
\end{frame}

%------------------------------------------------

\begin{frame}
\frametitle{Conjunction}
\begin{center}
\ig[width=10cm]{images/preattentive-conjunction.pdf}
\end{center}
\end{frame}

%------------------------------------------------

\begin{frame}
\frametitle{Theories of Preattentive Processing}

\bbi
  \item ``Feature Integration Theory'' by Anne Treisman.
  \item ``Texton Theory'' by Bela Julesz.
  \item ``Similarity'' Theory by Quinland and Humphreys.
  \item ``Guided Search'' Theory by Jeremy Wolfe.
  \item ``Boolean Map'' Theory by Huang et al.
\ei

\end{frame}

%------------------------------------------------

\begin{frame}
\begin{center}
\Huge{\hilit{Later Vision and \\ Postattentive Processing}}
\end{center}
\end{frame}

%------------------------------------------------

\begin{frame}
\frametitle{Postattentive Processing}

\bb{Preattentive processing asks in part:}
``What visual properties draw our eyes,
and therefore our focus of attention''.
\eb

\vspace{1cm}

\bb{Postattentive processing asks:}
``What happens to the visual representation of
an object when we stop attending to it (and look at something else)?''
\eb

\end{frame}

%------------------------------------------------

\begin{frame}
\frametitle{Later Vision: Pattern Perception}

\bbi
  \item Slow serial processing
  \item Involves working and long-term memory
  \item A combination of bottom-up feature processing and top-down attentional
  mechanisms
  \item Different pathways for object recognition and visually guided motion
\ei

\end{frame}

%------------------------------------------------

\begin{frame}
\frametitle{Ron Rensink's examples}
\begin{center}
\ig[width=11cm]{images/postattentive-airplane.pdf}
\end{center}
\end{frame}

%------------------------------------------------

\begin{frame}
\frametitle{Ron Rensink's examples}
\begin{center}
\ig[width=11cm]{images/postattentive-market.pdf}
\end{center}
\end{frame}

%------------------------------------------------

\begin{frame}
\frametitle{Ron Rensink's examples}
\begin{center}
\ig[width=11cm]{images/postattentive-corner.pdf}
\end{center}
\end{frame}

%------------------------------------------------

\begin{frame}
\frametitle{References}

\bbi
  \item \textbf{Perception in Visualization} by Christopher Healey.
  \item \textbf{The Functional Art} (chapter 6) by Alberto Cairo.
  \item \textbf{Visual Language for Designers} (principle 1) by Connie Malamed.
  \item \textbf{Information Dashboard Design} (chapter 4) by Stephen Few.
  \item \textbf{Interactive Data Visualization} (chapter 3) by Ward, Grinstein
  and Keim.
  \item \textbf{100 Things Every Designer Needs to Know About People} 
  (chapter 1) by Susan M. Weinschenk.
\ei

\end{frame}

%------------------------------------------------

\end{document}
