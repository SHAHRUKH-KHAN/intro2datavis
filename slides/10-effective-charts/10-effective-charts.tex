\documentclass[12pt]{beamer}\usepackage[]{graphicx}\usepackage[]{color}
%% maxwidth is the original width if it is less than linewidth
%% otherwise use linewidth (to make sure the graphics do not exceed the margin)
\makeatletter
\def\maxwidth{ %
  \ifdim\Gin@nat@width>\linewidth
    \linewidth
  \else
    \Gin@nat@width
  \fi
}
\makeatother

\definecolor{fgcolor}{rgb}{0.345, 0.345, 0.345}
\newcommand{\hlnum}[1]{\textcolor[rgb]{0.686,0.059,0.569}{#1}}%
\newcommand{\hlstr}[1]{\textcolor[rgb]{0.192,0.494,0.8}{#1}}%
\newcommand{\hlcom}[1]{\textcolor[rgb]{0.678,0.584,0.686}{\textit{#1}}}%
\newcommand{\hlopt}[1]{\textcolor[rgb]{0,0,0}{#1}}%
\newcommand{\hlstd}[1]{\textcolor[rgb]{0.345,0.345,0.345}{#1}}%
\newcommand{\hlkwa}[1]{\textcolor[rgb]{0.161,0.373,0.58}{\textbf{#1}}}%
\newcommand{\hlkwb}[1]{\textcolor[rgb]{0.69,0.353,0.396}{#1}}%
\newcommand{\hlkwc}[1]{\textcolor[rgb]{0.333,0.667,0.333}{#1}}%
\newcommand{\hlkwd}[1]{\textcolor[rgb]{0.737,0.353,0.396}{\textbf{#1}}}%
\let\hlipl\hlkwb

\usepackage{framed}
\makeatletter
\newenvironment{kframe}{%
 \def\at@end@of@kframe{}%
 \ifinner\ifhmode%
  \def\at@end@of@kframe{\end{minipage}}%
  \begin{minipage}{\columnwidth}%
 \fi\fi%
 \def\FrameCommand##1{\hskip\@totalleftmargin \hskip-\fboxsep
 \colorbox{shadecolor}{##1}\hskip-\fboxsep
     % There is no \\@totalrightmargin, so:
     \hskip-\linewidth \hskip-\@totalleftmargin \hskip\columnwidth}%
 \MakeFramed {\advance\hsize-\width
   \@totalleftmargin\z@ \linewidth\hsize
   \@setminipage}}%
 {\par\unskip\endMakeFramed%
 \at@end@of@kframe}
\makeatother

\definecolor{shadecolor}{rgb}{.97, .97, .97}
\definecolor{messagecolor}{rgb}{0, 0, 0}
\definecolor{warningcolor}{rgb}{1, 0, 1}
\definecolor{errorcolor}{rgb}{1, 0, 0}
\newenvironment{knitrout}{}{} % an empty environment to be redefined in TeX

\usepackage{alltt}
\usepackage{tikz}

% make it pretty
% get rid of junk
\usetheme{default}
\usefonttheme[onlymath]{serif}
\beamertemplatenavigationsymbolsempty

% define a bunch of colors
\definecolor{offwhite}{RGB}{255,250,240}
\definecolor{gray}{RGB}{155,155,155}
\definecolor{foreground}{RGB}{80,80,80}
\definecolor{background}{RGB}{255,255,255}
%\definecolor{title}{RGB}{255,199,0}
\definecolor{title}{RGB}{89,132,212}
%\definecolor{subtitle}{RGB}{89,132,212}
\definecolor{subtitle}{RGB}{255,199,0}
\definecolor{hilit}{RGB}{248,117,79}
\definecolor{vhilit}{RGB}{255,111,207}
\definecolor{lolit}{RGB}{200,200,200}
\definecolor{lit}{RGB}{255,199,0}
\definecolor{mdlit}{RGB}{89,132,212}
\definecolor{link}{RGB}{248,117,79}

% a few color macros
\newcommand{\hilit}{\color{hilit}}
\newcommand{\vhilit}{\color{vhilit}}
\newcommand{\lit}{\color{lit}}
\newcommand{\mdlit}{\color{mdlit}}
\newcommand{\lolit}{\color{lolit}}

% use those colors
\setbeamercolor{titlelike}{fg=title}
\setbeamercolor{subtitle}{fg=subtitle}
\setbeamercolor{frametitle}{fg=gray}
%\setbeamercolor{structure}{fg=subtitle}
\setbeamercolor{structure}{fg=title}
\setbeamercolor{institute}{fg=lolit}
\setbeamercolor{normal text}{fg=foreground,bg=background}
\setbeamertemplate{itemize subitem}{{\textendash}}
\setbeamerfont{itemize/enumerate subbody}{size=\small}
\setbeamerfont{itemize/enumerate subitem}{size=\small}

% center title of slides
\setbeamertemplate{blocks}[rounded]
\setbeamertemplate{frametitle}[default][center]

% page number
\setbeamerfont{page number in foot}{size=\footnotesize}
\setbeamertemplate{footline}[frame number]

% default link color
\hypersetup{colorlinks, urlcolor={link}}

% a few macros
\newcommand{\code}[1]{\texttt{#1}}
\newcommand{\hicode}[1]{{\hilit \texttt{#1}}}
\newcommand{\locode}[1]{{\lolit \texttt{#1}}}
\newcommand{\bb}[1]{\begin{block}{#1}}
\newcommand{\eb}{\end{block}}
\newcommand{\bi}{\begin{itemize}}
\newcommand{\bbi}{\vspace{4pt} \begin{itemize} \itemsep8pt}
\newcommand{\ei}{\end{itemize}}
\newcommand{\bv}{\begin{verbatim}}
\newcommand{\ev}{\end{verbatim}}
\newcommand{\ig}{\includegraphics}
\newcommand{\subt}[1]{{\footnotesize \color{subtitle} {#1}}}
\newcommand{\ttsm}{\tt \small}
\newcommand{\ttfn}{\tt \footnotesize}
\newcommand{\figh}[2]{\centerline{\includegraphics[height=#2\textheight]{#1}}}
\newcommand{\figw}[2]{\centerline{\includegraphics[width=#2\textwidth]{#1}}}



%------------------------------------------------

\title{Effective Charts}
\subtitle{Intro to Data Visualization}
\author{\href{http://www.gastonsanchez.com}{Gaston Sanchez}}
\institute{\href{https://creativecommons.org/licenses/by-sa/4.0/}{\tt \scriptsize \color{foreground} CC BY-SA 4.0}}
\date{}
\IfFileExists{upquote.sty}{\usepackage{upquote}}{}
\begin{document}

% no page number in first slide
{
  \setbeamertemplate{footline}{} 
  \frame{\titlepage} 
}

%------------------------------------------------

\begin{frame}
\begin{center}
\Huge{\hilit{Effective Charts}}
\end{center}
\end{frame}

%------------------------------------------------

\begin{frame}
\frametitle{Oppressing typography}
\begin{center}
\ig[height=7cm]{images/typography-oppress.pdf}
\end{center}
\end{frame}

%------------------------------------------------

\begin{frame}
\frametitle{Avoid highly stylized typography}
\begin{center}
\ig[height=7cm]{images/typography-stylized.pdf}
\end{center}
\end{frame}

%------------------------------------------------

\begin{frame}
\frametitle{Simple typography}
\begin{center}
\ig[height=7cm]{images/typography-simple.pdf}
\end{center}
\end{frame}

%------------------------------------------------

\begin{frame}
\frametitle{No need for 3D Effect}
\begin{center}
\ig[height=7cm]{images/3d-effect.pdf}
\end{center}
\end{frame}

%------------------------------------------------

\begin{frame}
\frametitle{Avoid distracting shades}
\begin{center}
\ig[height=7cm]{images/distracting-shades.pdf}
\end{center}
\end{frame}

%------------------------------------------------

\begin{frame}
\frametitle{Different colors for the same type of data?}
\begin{center}
\ig[height=7cm]{images/colors-same-data.pdf}
\end{center}
\end{frame}

%------------------------------------------------

\begin{frame}
\frametitle{Same color for the same type of data}
\begin{center}
\ig[height=7cm]{images/same-color-same-data.pdf}
\end{center}
\end{frame}

%------------------------------------------------

\begin{frame}
\frametitle{Truncated Baseline}
\begin{center}
\ig[height=7cm]{images/truncated-baseline.pdf}
\end{center}
\end{frame}

%------------------------------------------------

\begin{frame}
\frametitle{Zero Baseline}
\begin{center}
\ig[height=7cm]{images/zero-baseline.pdf}
\end{center}
\end{frame}

%------------------------------------------------

\begin{frame}
\frametitle{Darker Shades}
\begin{center}
\ig[width=10cm]{images/darker-shade-focal-point.pdf}
\end{center}
\end{frame}

%------------------------------------------------

\begin{frame}
\frametitle{Contrasting Color}
\begin{center}
\ig[width=10cm]{images/contrasting-color-negative-value.pdf}
\end{center}
\end{frame}

%------------------------------------------------

\begin{frame}
\frametitle{Background Color}
\begin{center}
\ig[width=10cm]{images/background-negative-zone.pdf}
\end{center}
\end{frame}

%------------------------------------------------

\begin{frame}
\frametitle{Avoid Zebra Pattern}
\begin{center}
\ig[width=10cm]{images/zebra-pattern.pdf}
\end{center}
\end{frame}

%------------------------------------------------

\begin{frame}
\frametitle{Sequential Shades}
\begin{center}
\ig[width=10cm]{images/lightest-darkest.pdf}
\end{center}
\end{frame}

%------------------------------------------------

\begin{frame}
\frametitle{Using Complementary Colors?}
\begin{center}
\ig[height=7cm]{images/careful-complementary-colors.pdf}
\end{center}
\end{frame}

%------------------------------------------------

\begin{frame}
\frametitle{Using Different Shades}
\begin{center}
\ig[height=7cm]{images/use-different-shades.pdf}
\end{center}
\end{frame}

%------------------------------------------------

\begin{frame}
\frametitle{Using Legends}
\begin{center}
\ig[height=7cm]{images/legends-right-sequence.pdf}
\end{center}
\end{frame}

%------------------------------------------------

\begin{frame}
\frametitle{Width of Lines}
\begin{center}
\ig[height=7cm]{images/lines-too-thin.pdf}
\end{center}
\end{frame}

%------------------------------------------------

\begin{frame}
\frametitle{Line Width}
\begin{center}
\ig[height=7cm]{images/lines-too-thick.pdf}
\end{center}
\end{frame}

%------------------------------------------------

\begin{frame}
\frametitle{Line Width}
\begin{center}
\ig[height=7cm]{images/ok-thickness.pdf}
\end{center}
\end{frame}

%------------------------------------------------

\begin{frame}
\frametitle{Avoid Spaghetti Lines}
\begin{center}
\ig[height=7cm]{images/spaghetti-lines.pdf}
\end{center}
\end{frame}

%------------------------------------------------

\begin{frame}
\frametitle{Facetting}
\begin{center}
\ig[width=10cm]{images/panel-of-charts.pdf}
\end{center}
\end{frame}

%------------------------------------------------

\begin{frame}
\frametitle{Dark Intensities}
\begin{center}
\ig[width=10cm]{images/line-dark-intensities.pdf}
\end{center}
\end{frame}

%------------------------------------------------

\begin{frame}
\frametitle{Dark Intensities}
\begin{center}
\ig[width=10cm]{images/line-emphasis.pdf}
\end{center}
\end{frame}

%------------------------------------------------

\begin{frame}
\frametitle{Avoid Long-distance Labeling}
\begin{center}
\ig[width=10cm]{images/long-distance-labeling.pdf}
\end{center}
\end{frame}

%------------------------------------------------

\begin{frame}
\frametitle{Direct Labeling}
\begin{center}
\ig[width=10cm]{images/direct-labeling.pdf}
\end{center}
\end{frame}

%------------------------------------------------

\begin{frame}
\frametitle{Avoid Long-distance Labeling}
\begin{center}
\ig[width=10cm]{images/long-distance-labeling2.pdf}
\end{center}
\end{frame}

%------------------------------------------------

\begin{frame}
\frametitle{Direct Labeling}
\begin{center}
\ig[width=10cm]{images/direct-labeling2.pdf}
\end{center}
\end{frame}

%------------------------------------------------

\end{document}
