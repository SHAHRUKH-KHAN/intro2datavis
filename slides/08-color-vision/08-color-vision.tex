\documentclass[12pt]{beamer}\usepackage[]{graphicx}\usepackage[]{color}
%% maxwidth is the original width if it is less than linewidth
%% otherwise use linewidth (to make sure the graphics do not exceed the margin)
\makeatletter
\def\maxwidth{ %
  \ifdim\Gin@nat@width>\linewidth
    \linewidth
  \else
    \Gin@nat@width
  \fi
}
\makeatother

\definecolor{fgcolor}{rgb}{0.345, 0.345, 0.345}
\newcommand{\hlnum}[1]{\textcolor[rgb]{0.686,0.059,0.569}{#1}}%
\newcommand{\hlstr}[1]{\textcolor[rgb]{0.192,0.494,0.8}{#1}}%
\newcommand{\hlcom}[1]{\textcolor[rgb]{0.678,0.584,0.686}{\textit{#1}}}%
\newcommand{\hlopt}[1]{\textcolor[rgb]{0,0,0}{#1}}%
\newcommand{\hlstd}[1]{\textcolor[rgb]{0.345,0.345,0.345}{#1}}%
\newcommand{\hlkwa}[1]{\textcolor[rgb]{0.161,0.373,0.58}{\textbf{#1}}}%
\newcommand{\hlkwb}[1]{\textcolor[rgb]{0.69,0.353,0.396}{#1}}%
\newcommand{\hlkwc}[1]{\textcolor[rgb]{0.333,0.667,0.333}{#1}}%
\newcommand{\hlkwd}[1]{\textcolor[rgb]{0.737,0.353,0.396}{\textbf{#1}}}%
\let\hlipl\hlkwb

\usepackage{framed}
\makeatletter
\newenvironment{kframe}{%
 \def\at@end@of@kframe{}%
 \ifinner\ifhmode%
  \def\at@end@of@kframe{\end{minipage}}%
  \begin{minipage}{\columnwidth}%
 \fi\fi%
 \def\FrameCommand##1{\hskip\@totalleftmargin \hskip-\fboxsep
 \colorbox{shadecolor}{##1}\hskip-\fboxsep
     % There is no \\@totalrightmargin, so:
     \hskip-\linewidth \hskip-\@totalleftmargin \hskip\columnwidth}%
 \MakeFramed {\advance\hsize-\width
   \@totalleftmargin\z@ \linewidth\hsize
   \@setminipage}}%
 {\par\unskip\endMakeFramed%
 \at@end@of@kframe}
\makeatother

\definecolor{shadecolor}{rgb}{.97, .97, .97}
\definecolor{messagecolor}{rgb}{0, 0, 0}
\definecolor{warningcolor}{rgb}{1, 0, 1}
\definecolor{errorcolor}{rgb}{1, 0, 0}
\newenvironment{knitrout}{}{} % an empty environment to be redefined in TeX

\usepackage{alltt}
\usepackage{tikz}

% get rid of junk
\usetheme{default}
\usefonttheme[onlymath]{serif}
\beamertemplatenavigationsymbolsempty

% define a bunch of colors
\definecolor{offwhite}{RGB}{255,250,240}
\definecolor{gray}{RGB}{155,155,155}
\definecolor{foreground}{RGB}{80,80,80}
\definecolor{background}{RGB}{255,255,255}
%\definecolor{title}{RGB}{255,199,0}
\definecolor{title}{RGB}{89,132,212}
%\definecolor{subtitle}{RGB}{89,132,212}
\definecolor{subtitle}{RGB}{255,199,0}
\definecolor{hilit}{RGB}{248,117,79}
\definecolor{vhilit}{RGB}{255,111,207}
\definecolor{lolit}{RGB}{200,200,200}
\definecolor{lit}{RGB}{255,199,0}
\definecolor{mdlit}{RGB}{89,132,212}
\definecolor{link}{RGB}{248,117,79}

% a few color macros
\newcommand{\hilit}{\color{hilit}}
\newcommand{\vhilit}{\color{vhilit}}
\newcommand{\lit}{\color{lit}}
\newcommand{\mdlit}{\color{mdlit}}
\newcommand{\lolit}{\color{lolit}}

% use those colors
\setbeamercolor{titlelike}{fg=title}
\setbeamercolor{subtitle}{fg=subtitle}
\setbeamercolor{frametitle}{fg=gray}
%\setbeamercolor{structure}{fg=subtitle}
\setbeamercolor{structure}{fg=title}
\setbeamercolor{institute}{fg=lolit}
\setbeamercolor{normal text}{fg=foreground,bg=background}
\setbeamertemplate{itemize subitem}{{\textendash}}
\setbeamerfont{itemize/enumerate subbody}{size=\small}
\setbeamerfont{itemize/enumerate subitem}{size=\small}

% center title of slides
\setbeamertemplate{blocks}[rounded]
\setbeamertemplate{frametitle}[default][center]

% page number
\setbeamerfont{page number in foot}{size=\footnotesize}
\setbeamertemplate{footline}[frame number]

% default link color
\hypersetup{colorlinks, urlcolor={link}}

% a few macros
\newcommand{\code}[1]{\texttt{#1}}
\newcommand{\hicode}[1]{{\hilit \texttt{#1}}}
\newcommand{\locode}[1]{{\lolit \texttt{#1}}}
\newcommand{\bb}[1]{\begin{block}{#1}}
\newcommand{\eb}{\end{block}}
\newcommand{\bi}{\begin{itemize}}
\newcommand{\bbi}{\vspace{4pt} \begin{itemize} \itemsep8pt}
\newcommand{\ei}{\end{itemize}}
\newcommand{\bv}{\begin{verbatim}}
\newcommand{\ev}{\end{verbatim}}
\newcommand{\ig}{\includegraphics}
\newcommand{\subt}[1]{{\footnotesize \color{subtitle} {#1}}}
\newcommand{\ttsm}{\tt \small}
\newcommand{\ttfn}{\tt \footnotesize}
\newcommand{\figh}[2]{\centerline{\includegraphics[height=#2\textheight]{#1}}}
\newcommand{\figw}[2]{\centerline{\includegraphics[width=#2\textwidth]{#1}}}



%------------------------------------------------

\title{Color Vision}
\subtitle{Intro to Data Visualization}
\author{\href{http://www.gastonsanchez.com}{Gaston Sanchez}}
\institute{\href{https://creativecommons.org/licenses/by-sa/4.0/}{\tt \scriptsize \color{foreground} CC BY-SA 4.0}}
\date{}
\IfFileExists{upquote.sty}{\usepackage{upquote}}{}
\begin{document}


{
  \setbeamertemplate{footline}{} % no page number here
  \frame{
    \titlepage
  } 
}

%------------------------------------------------

\begin{frame}
\begin{center}
\Huge{\hilit{Light and Color}}
\end{center}
\end{frame}

%------------------------------------------------

\begin{frame}
\frametitle{Light Recap}

\bbi
  \item Light is a form of electromagnetic radiation.
  \item Electromagnetic radiations are characterized by their wavelength.
  \item Visible light has wavelengths in a narrow band centered on 600 nanometers.
\ei

\end{frame}

%------------------------------------------------

\begin{frame}
\frametitle{Electromagnetic Spectrum}
\begin{center}
\ig[width=10cm]{images/visible-spectrum.pdf}
\end{center}
\end{frame}

%------------------------------------------------

\begin{frame}
\frametitle{Visible Light}

\bb{Visible Spectrum}
If the electromagnetic spectrum spanned the distance from Los Angeles to New 
York City, the part visible to the human eye would span the width of a dime.
\eb

\end{frame}

%------------------------------------------------

\begin{frame}
\frametitle{Visible Light}

\bi
  \item Our eyes detect wavelengths in a tiny portion of the \textit{EM} spectrum.
  \item We call this the \textit{visible light spectrum}.
  \item We perceive short wavelengths as blue.
  \item We perceive longer wavelengths as red.
  \item We cannot perceive wavelength beyond the limits of the visible spectrum.
  \item Shorter wavelengths of ultraviolet light.
  \item Longer wavelengths of infrarred radiation.
\ei

\end{frame}

%------------------------------------------------

\begin{frame}
\frametitle{Electromagnetic Spectrum}
\begin{center}
\ig[width=9cm]{images/visible-spectrum.jpg}

{\lolit www.astronomersgroup.org/EMspectrum.html}
\end{center}
\end{frame}

%------------------------------------------------

\begin{frame}
\frametitle{Visible Spectrum}
\begin{center}
\ig[width=9cm]{images/spectrum-thirds.pdf}

If the visible spectrum is divided into thirds, the predominant colors are blue, green, and red.
\end{center}
\end{frame}

%------------------------------------------------

\begin{frame}
\frametitle{Some considerations}

\bbi
  \item Color is part of how we sense the world around us.
  \item Light enters the eyes.
  \item It is processed by light receptors (cones and rods).
  \item And sent via the optic nerves to the brain for further processing and
  interpretation.
  \item Light varies in wavelengths, which our eyes and brain interpret as
  varying colors.
\ei

\end{frame}
  
%------------------------------------------------

\begin{frame}
\frametitle{}
\begin{center}
\ig[height=8cm]{images/larson-cartoon.pdf}
\end{center}
\end{frame}

%------------------------------------------------

\begin{frame}
\begin{center}
\Huge{\hilit{Color Vision Theories}}
\end{center}
\end{frame}

%------------------------------------------------

\begin{frame}
\frametitle{Color Vision Theories}

There are two complementary color vision theories that explain how we perceive 
colors, and how our brain ``sees'' colors

\bi
  \item \textbf{Trichromatic} theory.
  \item \textbf{Opponent} theory.
\ei

\end{frame}

%------------------------------------------------

\begin{frame}
\begin{center}
\Huge{\hilit{Trichromatic Theory}}
\end{center}
\end{frame}

%------------------------------------------------

\begin{frame}
\frametitle{Trichromatic Theory}

\bb{Young-Helmholtz Theory}
\bbi
  \item Originally proposed by Thomas Young in 1802.
  \item Further expanded by Hermann von Helmholtz.
  \item Human color vision could be explained by the existence of 3 receptors: 
  Red, Green, and Blue.
  \item {\scriptsize \url{https://en.wikipedia.org/wiki/Young\%E2\%80\%93Helmholtz_theory}}
\ei
\eb

\end{frame}

%------------------------------------------------

\begin{frame}
\frametitle{3 kinds of light receptors}
\begin{center}
\ig[width=10cm]{images/young-helmholtz.pdf}
\end{center}
\end{frame}

%------------------------------------------------

\begin{frame}
\frametitle{Trichromatic Theory}

\bbi
  \item Young suggested that the eye contained different photoreceptor cells.
  \item Each cell was sensitive to different wavelengths of light in the visible spectrum.
  \item Young's original theory was based on Red, Green, and Blue receptors.
  \item We now know that there are really three kinds of color receptors in the eye.
  \item As a result the 3 kinds of cones in the eye are called \textbf{R}, 
  \textbf{G}, and \textbf{B}.
\ei


\end{frame}

%------------------------------------------------

\begin{frame}
\frametitle{Retina Cell Receptors}
\begin{center}
\ig[width=11cm]{images/rods-and-cones.pdf}
\end{center}
\end{frame}

%------------------------------------------------

\begin{frame}
\frametitle{Trichromatic Theory}

\bb{Young-Helmholtz Theory}
Helmholtz used color-matching experiments where participants would alter the 
amounts of three different wavelengths of light to match a test color.

\vspace{1cm}

Participants could not match the colors if they used only two wavelengths, 
but could match any color in the spectrum if they used three (Red, Green, Blue).
\eb

\end{frame}

%------------------------------------------------

\begin{frame}
\begin{center}
\Huge{\hilit{Opponent Theory}}
\end{center}
\end{frame}

%------------------------------------------------

\begin{frame}
\frametitle{Opponent Theory}

\bb{Opponent-Process Theory}
\bbi
  \item Predates Young-Helmholtz theory. Dates as far back as Leonardo Da Vinci.
  \item The color pairs red-green and yellow-blue are in opposition.
  \item No color can simultaneously exhibit both redness and grenness, 
  or blueness and yellowness.
\ei
\eb

\end{frame}

%------------------------------------------------

\begin{frame}
\frametitle{Opponent Theory}

\bb{Opponent-Process Theory}
\bbi
  \item Formally developed by Ewald Hering.
  \item Hering noted color combinations that we never see \\
  {\lolit reddish-green or yellowish-blue}
  \item This theory suggests that color perception is controlled by the 
  activity of three opponent systems: \\
  white and black, blue and yellow, and red and green.
  \item {\scriptsize \url{https://en.wikipedia.org/wiki/Opponent-process_theory}}
\ei
\eb

\end{frame}

%------------------------------------------------

\begin{frame}
\frametitle{Opponent Theory}

\bb{Opponent-Process Theory}
Colors are created by the eye and brain as combinations of red-green, and 
yellow-blue color signals.
\eb

\bb{R-G and B-Y pairs}
Our eyes and brain create four basic colors: yellow, blue, red, and green, 
arranged in two pairs.
\eb

\end{frame}

%------------------------------------------------

\begin{frame}
\frametitle{Color Pairs of Opponent Theory}
\begin{center}
\ig[height=7cm]{images/opponent-colors.pdf}
\end{center}
\end{frame}

%------------------------------------------------

\begin{frame}
\frametitle{Opponent Theory}

\bb{A color can be described by three parameters}
\bbi
  \item Where it lies on a light/dark scale
  \item Where it lies on a red/green scale
  \item Where it lies on a blue/yellow scale
\ei
\eb

These three dimensions, ``green or red,'', ``blue or yellow,'' and 
brightness, form the foundation for the way our brain perceives color.

\end{frame}

%------------------------------------------------

\begin{frame}
\frametitle{Opponent Theory}

Our sensation of color comes from nerve cells that send messages to the brain about:

\bi
  \item The brightness of color.
  \item Greenness vs. redness. 
  \item Blueness vs. yellowness.
  \item Color nerves sense green or red, but never both.
  \item Likewise color nerves sense blue or yellow, but never both. 
  \item We never see bluish-yellows or reddish-greens.
\ei

\end{frame}

%------------------------------------------------

\begin{frame}
\frametitle{Color Opponency}
\begin{center}
\ig[width=11cm]{images/color-opponency.pdf}

{\scriptsize {\lolit C. Ware, ``Visual Thinking for Design''}}
\end{center}
\end{frame}

%------------------------------------------------

\begin{frame}
\frametitle{Opponent Colors and Television}

\bbi
  \item Initially, television was only available in black and white.
  \item Later on, technology became available to make color television.
  \item Engineers faced the problem of how to transmit the color information but still remain compatible with all the existing black and white sets.
  \item They chose to add the color information by adding two additional color signals.
  \item The two signals were the color positions on the opponent red/green and yellow/blue scales.
\ei

\end{frame}

%------------------------------------------------

\begin{frame}
\begin{center}
\Huge{\hilit{Evolution of Color Vision}}
\end{center}
\end{frame}

%------------------------------------------------

\begin{frame}
\frametitle{Evolution of Color Vision}

\bbi
  \item The human color vision system appears to have evolved in steps.
  \item Initially our (very remote) ancestors had a single class of light sensitive cells.
  \item At a point predating the evolution of the mammals this single class of cells differentiated into separate yellow and blue classes.
  \item Much later, in primates, the class of yellow sensitive cells differentiated into separate red and green sensitive classes.
\ei

\end{frame}

%------------------------------------------------

\begin{frame}
\frametitle{Primitive Color}
\begin{center}
\ig[width=10cm]{images/primitive-color.pdf}

{\scriptsize {\lolit image as it would appear to an animal with just 
a yellow/blue based vision system}}
\end{center}
\end{frame}

%------------------------------------------------

\begin{frame}
\frametitle{Primate Color}
\begin{center}
\ig[width=10cm]{images/primate-color.pdf}

{\scriptsize {\lolit image as it would appear to a primate with 
a red/green/blue based vision system}}
\end{center}
\end{frame}

%------------------------------------------------

\begin{frame}
\frametitle{Primate Color Vision: Searching for Fruits}
\begin{center}
\ig[width=11cm]{images/primate-color-vision-fruits.pdf}
\end{center}
\end{frame}

%------------------------------------------------

\begin{frame}
\frametitle{Fruits and Primate Color Vision}

\bbi
  \item Natural task facing monkeys foraging for fruit (\textit{a}).
  \item Typical stimulus array from laboratory: searching for orange circle (\textit{b}).
  \item Photograph as it would appear to a protanope (\textit{c}).
  \item Photograph as it would appear to a deuteranope (\textit{d}).
\ei

{\scriptsize {\lolit Regan et al. (2001) Fruits, foliage and the evolution of primate colour vision. The Philos Trans R Soc Lond B Biol Sci. 356, 229-283.}}

\end{frame}

%------------------------------------------------

\begin{frame}
\begin{center}
\Huge{\hilit{Trichromatic and Opponent Theories}}
\end{center}
\end{frame}

%------------------------------------------------

\begin{frame}
\frametitle{Trichromatic and Opponent Theories}

\bi
  \item Both theories are complementary
  \item The Thrichromatic theory applies in the eyes (physical-sensory)
  \item The Opponent theory applies in the brain (cognition)
\ei

\end{frame}

%------------------------------------------------

\begin{frame}
\frametitle{Trichromatic and Opponent Theories}

\bb{Color vision in three major processes}
\bbi
  \item \textbf{Trichromatic input}: information is recorded by the responses of 
  the L, M and S cone cells in the retina.
  \item \textbf{Opponent output}: responses from the L, M and S cones are converted 
  into signals for yellowness vs blueness and redness vs greenness, plus total brightness.
  \item \textbf{Processing for color constancy}: color information from the visual field is analyzed, interpreting object properties (i.e. hue, value and chroma) and lighting properties (i.e. hue, brightness and saturation of the illumination).
\ei
\eb

\end{frame}


%------------------------------------------------

\begin{frame}
\frametitle{References}

\bbi
  \item \url{http://www.handprint.com/HP/WCL/color7.html} \\
  by Bruce MacEvoy
  \item \textbf{The Dimensions of Colour} by David Briggs
  \item \textbf{Causes of Color} by WebExhibits \\
  {\scriptsize \url{http://www.webexhibits.org/causesofcolor/1.html}}
  \item \textbf{How do we perceive color} by ColoRotate \\
  {\tiny \url{http://learn.colorotate.org/how-do-we-perceive-color}}
\ei

\end{frame}

%------------------------------------------------

\end{document}
